\chapter{Introdução}

\section{Introdução}

Este presente trabalho tem como objectivo demonstrar a implementação de um sistema de gestão de receitas e ingredientes para um Chefe de Restaurante.

Sendo este uma \textit{fork} de duas tarefas de quatro totais demonstradas num trabalho prévio, o qual no capitulo seguinte está uma renderização do mesmo.

\section{Estrutura deste Relatório}

Este relatório está organizado da seguinte forma: Introdução, Analise do Sistema (onde se usa o trabalho de analise da época normal, o qual está bom e saturado de boa informação), Inicio e caracterização do Projecto, Base de dados, Implementação da API, Implementação da WebApp, Testes, Conclusão.

\section{Análise do Sistema}

A análise deste sistema foi feita em época normal, para não refazer (e possivelmente de uma pior forma) esse trabalho anterior, decidi incluir uma renderização desse \textit{.pdf} na secção dos Anexos, \hyperref[pdf:analysis]{nomeadamente o primeiro}. No entanto podemos resumir.

\subsection{Análise do Sistema Resumo}

O objectivo geral deste trabalho é a criação de uma aplicação web que permita o acesso aos clientes e funcionários do restaurante aos menus disponíveis na ementa, mesas disponíveis, reservas no restaurante, gestão do stock de alimentos.

A análise apresenta três actores: João Miguel (Cliente), Ivone Silva (Gerente) e Raul Joaquim (Funcionário). Os quais são apresentados os detalhes, vida e casos de uso para cada persona.

Sendo que existem no total quinze principais acções descritas na análise, as quais são agrupas em quatro casos de uso: Gestão de Receitas, Gestão de Ingredientes e Stock, Gestão de Reservas e Mesas, Acções de cliente (consulta de receitas e reservas). Seguidamente é apresentado os gráficos UML e os \textit{drafts} das interfaces referentes aos casos de uso. 

É através dessas necessidades que é criada e apresentado os modelos de Entidade-Relação e da Base de Dados. Os quais são imediatamente apresentados nesse capítulo.

\subsection{Casos de Uso a mim atribuídos}

Dos quatro casos de uso, foram-me atribuídos os: Gestão de Receitas, Gestão de Ingredientes e Stock. Os quais facilita, caso não haja tempo para implementar sistema de Login, pode ser implementado como uma aplicação direcionada ao staff de cozinha.

Neste relatório descrevo assim essa versão direcionada aos proficionais de cozinha, criada em ASP.NET Core.