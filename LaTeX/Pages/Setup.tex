\chapter{\textit{Setup}}

Aqui descrevo os passos para criar um projecto ASP.NET Core (\textit{Database First}).

\section{Pré-requisitos}

Instalar o Visual Studio 2019 com ASP.NET Core \textit{Development} e \textit{Database  Development Tools}, de seguida instalamos o \textit{SQL Server Express} e o \textit{SQL Server Management Studio}.

Abrir o terminal e executar o comando:

\begin{itemize}

  \item \texttt{dotnet\ tool\ install\ -\/-global\ dotnet-ef}
\end{itemize}

\section{Pacotes}

Instalar pacotes de NuGet:

\begin{itemize}

  \item Microsoft.AspNetCore.Identity.EntityFrameworkCore
  \item Microsoft.AspNetCore.Mvc.Razor.RuntimeCompilation
  \item Microsoft.EntityFrameworkCore.SqlServer
  \item Microsoft.EntityFrameworkCore.Tools
  \item Microsoft.VisualStudio.Web.CodeGeneration.Design Atenção:
  \item Estou a usar o Visual Studio 2019, logo só tenho a versão do .NET Core 5.0. Como tal:
  \item As versões dos pacotes tem de ser 5.0.XX para funcionar.
\end{itemize}

\section{SQL Server}

Criar uma base de dados:

\begin{itemize}

  \item Criar uma base de dados \textit{SQL Server} via \textit{Queries} no \textit{SQL Server Management Studio}.
  \item Todas as tabelas devem ter PKs, devem ser NOT NULL e as FKs têm de fazer ON DELETE CASCADE. O \textit{EF Core} não suporta Tabelas sem PKs.
\end{itemize}

\section{Scaffolding}

Abrir o terminal (\textit{CTRL+Ç}) e executar o comando:

\begin{itemize}
  \item \texttt{dotnet\ ef\ dbcontext\ scaffold\ "Server=.\\SQLExpress;Database=DatabaseName.dbo;Trusted\_Connection=True;"}\\\texttt{\ Microsoft.EntityFrameworkCore.SqlServer\ -o Models -f}
\end{itemize}

Entrar em todos os ficheiros gerados e eliminar todos os \textit{partial},
\textit{virtual} e \textit{?} que não são necessários para um funcionamento correcto, em especifico, de
seguida, entrar no ficheiro \textit{ProjectNameContext.cs} e remover a menção do
\textit{partial class final} e a sua \textit{override implementation}.
